\documentclass{article}
\usepackage[utf8]{inputenc}

\title{Context aware convolutional neural network using self-attention mechanism }
% \author{Mahmoud Zaky Fetoh}
\date{November 2022}

\begin{document}

\maketitle


\section{Executive Summary}

Brain Tumor is an abnormal growth of the brain cells\cite{deangelis2001brain}. Brain tumor leads to brain cancer and death\cite{deangelis2001brain}. Computer aided diagnosis (CAD) tools assists clinician to make a brain Tumor diagnosis.
~Deep learning \cite{lecun2015deep} models, namely Convolutional Neural Network (CNN) \cite{lecun1989handwritten}, have shown a great performance in computer vision problems including object detection\cite{erhan2014scalable}\cite{girshick2014rich}\cite{sermanet2013overfeat}\cite{redmon2016you}, object recognition\cite{simonyan2014very}\cite{he2016deep} and many others.
This study proposes a Deep learning Network that exploit self-attention mechanism \cite{vaswani2017attention} to capture context relationship between extracted feature. Features are extracted using a stack of convolutional layers that exploit a high transformation cardinality\cite{xie2017aggregated} on top of learned residual mapping\cite{he2016deep}.


\section{Survey of Background Literature}

\cite{abiwinanda2019brain} proposes an architecture that is simpler than current a state-of-the-art classification networks. Authors have used 3064 T-1 weighted
CE-MRI of brain tumor images dataset\cite{cheng2017brain}. They achieve a training
and validation accuracies of architecture 2 at best is 98.51\% and 84.19\%, respectively. Their architecture suffer from severe overfitting.\\
\cite{badvza2020classification} developed network is simpler than already-existing pre-trained networks, and it was tested on T1-weighted contrast-enhanced magnetic resonance images\cite{cheng2017brain}.  10-fold cross-validation method was obtained for the record-wise cross-validation for the augmented data set, and, in that case, the accuracy was 96.56\%. Resdual learning\cite{he2016deep} can enhance the performance and reduce the train time However, it is not applied.\\
\cite{raza2022hybrid} fine-tuned googLeNet\cite{szegedy2015going} network by replacing the last 5 layer with a new 15 convolutional layer. Thair method ware tested on T1-weighted contrast-enhanced magnetic resonance images\cite{cheng2017brain}. Thair method achieved  99.67\% accuracy, 99.6\% precision, 100\% recall, and a 99.66\% F1-score. 





\subsection{Overview}

Give Context and brief argument for the need for progress in this research area and what you propose to do, in simple, non-technical terms (for the benefit of non-technical experts). Also, briefly emphasize the impact and outcomes from thus work in tangible terms, such as societal benefits
\subsection{Background}

Give a thorough introduction to the background science relevant to the general area of the proposed research (Much of this come from the review). Explain any relevant theory and standard techniques used in the area, along with any acronyms and jargon typical to the field. Also provide background material on any new methodologies if you intend to introduce to the field. Remember that it is always better avoid the use of abbreviations and acronyms as much as possible

\subsection{Relevance/Impact}

In the context provided by the background science, layout the argument that justifies the need for research in this area. Point out the significance of these works to the field, and practice. Comment on its potential impact on the development of the scientific area and the society/sector as a whole. What are we going to learn? Why is it important to learn this? How will this new knowledge translate benefits to the sector/society such as economic impact, improvement in quality of life, career, improvement in environment, improvement and closing the gender gap in ICT ,. etc etc...

\section{Proposed Methodology}
Outline the experimental and theoretical methods specific to the proposed research. Justify the choice of methods as opposed to alternatives. Avoid giving too much detailed information; just outline the general approach and why you chose to use them

\section{Research Plan}
State the long - and short -term objectives of your research program. Outline specific projects planned to meet these goals, including the timelines for completion of each stage, methods to be used, and dissemination of results. 

\section{Resources}
Provide details on the instrumentation and materials needed, along with the estimates of human resources required for each project/activity throughout the lifetime of the project

\bibliographystyle{plain}
\bibliography{mybib}

\end{document}